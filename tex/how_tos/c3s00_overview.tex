

% \section{综述 Overview}\label{ux7efcux8ff0-overview}

综述 Overview

\subsection{Variables:
创建,初始化,保存,和恢复}\label{variables-ux521bux5efaux521dux59cbux5316ux4fddux5b58ux548cux6062ux590d}

TensorFlow Variables 是内存中的容纳 tensor
的缓存。这一小节介绍了用它们在模型训练时(during
training)创建、保存和更新模型参数(model parameters) 的方法。

\href{../how_tos/variables.md}{参看教程}

\subsection{TensorFlow 机制 101}\label{tensorflow-ux673aux5236-101}

用 MNIST 手写数字识别作为一个小例子,一步一步的将使用 TensorFlow
基础架构(infrastructure)训练大规模模型的细节做详细介绍。

\href{../tutorials/mnist_tf.md}{参看教程}

\subsection{TensorBoard:
学习过程的可视化}\label{tensorboard-ux5b66ux4e60ux8fc7ux7a0bux7684ux53efux89c6ux5316}

对模型进行训练和评估时,TensorBoard
是一个很有用的可视化工具。此教程解释了创建和运行 TensorBoard
的方法,和使用摘要操作(Summary ops)的方法,通过添加摘要操作(Summary
ops),可以自动把数据传输到 TensorBoard 所使用的事件文件。

\href{../how_tos/summaries_and_tensorboard.md}{参看教程}

\subsection{TensorBoard:
图的可视化}\label{tensorboard-ux56feux7684ux53efux89c6ux5316}

此教程介绍了在 TensorBoard
中使用可视化工具的方法,它可以帮助你理解张量流图的过程并 debug。

\href{../how_tos/graph_viz.md}{参看教程}

\subsection{数据读入}\label{ux6570ux636eux8bfbux5165}

此教程介绍了把数据传入 TensorSlow 程序的三种主要的方法: Feeding,
Reading 和 Preloading.

\href{../how_tos/reading_data.md}{参看教程}

\subsection{线程和队列}\label{ux7ebfux7a0bux548cux961fux5217}

此教程介绍 TensorFlow
中为了更容易进行异步和并发训练的各种不同结构(constructs)。

\href{../how_tos/threading_and_queues.md}{参看教程}

\subsection{添加新的 Op}\label{ux6dfbux52a0ux65b0ux7684-op}

TensorFlow 已经提供一整套节点操作()operation),你可以在你的 graph
中随意使用它们,不过这里有关于添加自定义操作(custom op)的细节。

\href{../how_tos/adding_an_op.md}{参看教程}。

\subsection{自定义数据的
Readers}\label{ux81eaux5b9aux4e49ux6570ux636eux7684-readers}

如果你有相当大量的自定义数据集合,可能你想要对 TensorFlow 的 Data
Readers 进行扩展,使它能直接以数据自身的格式将其读入。

\href{../how_tos/new_data_formats.md}{参看教程}。

\subsection{使用 GPUs}\label{ux4f7fux7528-gpus}

此教程描述了用多个 GPU 构建和运行模型的方法。

\href{../how_tos/using_gpu.md}{参看教程}

\subsection{共享变量 Sharing
Variables}\label{ux5171ux4eabux53d8ux91cf-sharing-variables}

当在多 GPU 上部署大型的模型,或展开复杂的 LSTMs 或 RNNs
时,在模型构建代码的不同位置对许多相同的变量(Variable)进行读写常常是必须的。设计变量作用域(Variable
Scope)机制的目的就是为了帮助上述任务的实现。

\href{../how_tos/variable_scope/index.md}{参看教程}。

原文: \href{http://tensorflow.org/how_tos/index.html}{How-to}

翻译:\href{https://github.com/TerenceCooper}{Terence Cooper}

校对:\href{https://github.com/lonlonago}{lonlonago}